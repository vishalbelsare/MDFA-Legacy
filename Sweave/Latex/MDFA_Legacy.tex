% -*- mode: noweb; noweb-default-code-mode: R-mode; -*-
\documentclass[a4paper]{book}
%\documentclass[envcountsame,envcountchap]{svmono}


\usepackage{graphicx,url}
\usepackage{amssymb}

\def\pf{{\bf Proof. }}
\def\logimplies{\Rightarrow}
\def\convinlaw{\stackrel{{\cal L}}{\Longrightarrow }}
\def\convinp{\stackrel{P}{\longrightarrow }}
\def\convas{\stackrel{a.s.}{\longrightarrow }}
\def\convv{\stackrel{v}{\longrightarrow}}
\def\asymp{\stackrel{{\mathbb P}}{\sim}}
\def\RR{\mathbb R}
\def\ZZ{\mathbb Z}
\def\QQ{\mathbb Q}
\def\NN{\mathbb N}
\def\MM{\mathbb M}
\def\LL{\mathbb L}
\def\EE{\mathbb E}
\def\PP{\mathbb P}
\def\DD{\mathbb D}
\def\WW{\mathbb W}
\def\FF{\mathbb F}
\def\II{\mathbb I}
\def\FF{\mathbb F}
\def\XX{\mathbb X}
\def\CC{\mathbb C}
\def\sige{\sigma_{\epsilon}}
\def\ttheta{\widetilde{\theta}}
\def\tTheta{\widetilde{\Theta}}
\def\tsig{\widetilde{\sigma}^2}
\def\tc{\widetilde{c}}
\def\etheta{\widehat{\theta}}
\def\eTheta{\widehat{\Theta}}
\def\esig{\widehat{\sigma}^2}
\def\ptheta{\underline{\theta}}
\def\pTheta{\underline{\Theta}}
\def\psig{\underline{\sigma}^2}

\def\eqinlaw{\stackrel{{\cal L}}{=}}
\def\tends{\rightarrow}
\def\tendsinf{\rightarrow\infty}
\def\isodynamo{\Leftrightarrow}

\newtheorem{Theorem}{Theorem}
\newtheorem{Lemma}{Lemma}
\newtheorem{Corollary}{Corollary}
\newtheorem{Proposition}{Proposition}
\newtheorem{Definition}{Definition}
\newtheorem{Remark}{Remark}
\newtheorem{Example}{Example}
\newtheorem{Exercise}{Exercise}
\newtheorem{Illustration}{Illustration}
\newcommand{\mbf}[1]{\mbox{\boldmath $#1$}}

\setlength{\textwidth}{6.5in} \setlength{\textheight}{9in}
\setlength{\evensidemargin}{12pt} \setlength{\oddsidemargin}{0in}
\setlength{\topmargin}{1in}
\renewcommand{\baselinestretch}{1.3}
\setlength{\headheight}{0.2in} 
\setlength{\headsep}{0.2in}

%- Makes the section title start with Appendix in the appendix environment
\newcommand{\Appendix}
{%\appendix
%\def\thesection{Appendix~\Alph{section}}
\def\thesection{Appendix~\Alph{chapter}}
%\def\thesubsection{\Alph{section}.\arabic{subsection}}
%\def\thesubsection{A.\arabic{subsection}}
\def\thesubsection{A.\arabic{section}}
}


%\pagestyle{empty}
\usepackage{amssymb}
\usepackage{amsmath}
\usepackage{latexsym}
\usepackage{epsfig}
%\usepackage{html}
\usepackage{verbatim}
\usepackage{hyperref}
\usepackage{float}
\usepackage[utf8]{inputenc}
\usepackage{a4wide}

\title{Multivariate Real-Time Signal Extraction}
\author{Marc Wildi and Tucker McElroy}




\usepackage{Sweave}
\begin{document}

\maketitle

\date{}

%\SweaveOpts{prefix.string=c:/wia/tmp/bar}

\frontmatter%%%%%%%%%%%%%%%%%%%%%%%%%%%%%%%%%%%%%%%%%%%%%%%%%%%%%%

%\include{dedic}
%\newpage
%\phantom{rete}
%\newpage
%\include{preface}




\tableofcontents


\mainmatter%%%%%%%%%%%%%%%%%%%%%%%%%%%%%%%%%%%%%%%%%%%%%%%%%%%%%%%

%-----------------------------------------------

% Chapter 1
% Parallelized computation chapters customization and replication
% simanz<-100 chapters customization and replication
% Load all chapters

\chapter{Introduction}\label{intro_sec}

\section{Overview}

\subsection{Signals and Extraction}

In the applications of time series analysis to macroeconomics, finance, and quality control
 it is essential to extract useful information about trends, turning points, and anomalies
 in real time.  The practitioner does not have the luxury of sifting past data for 
 structural breaks, indicators of regime change, or changes to volatility.  Informative elections are
 contingent upon understanding the dynamics of various time series at time present.  Because
 long-term movements, as well as aberrations, are defined in terms of the long-run behavior of a 
 time series over past, present, and future, any analysis of the present state necessarily involves
 a degree of forecasting.  This  broad topic is referred to as real-time signal extraction.

A signal is any component of a time series that is deemed useful for a particular application.  
If long-term movements are of interest, the signal is a trend.  If short-term fluctuations about
 a longer-term mean are of interest, the signal is a cycle.  If shocks, due to rare
 terrorist events or natural disasters, are of interest, the signal consists of the extreme values.
 If regular patterns of an annual period, linked to cultural or meteorological patterns, are of interest,
 the signal is a seasonal component.

However, these signals are not directly observable at time present, because in each case their
 definition involves all the past and future values of a time series -- but the future is unknown,
 and only part of the past is available to us.  The statistical processes by which a signal is
 estimated from available data is referred to as extraction, and the residual from the 
 signal extraction is referred to as the noise.  Whereas signals can be estimated from historical, or past,
 sections of a time series, when effort is focused upon time present we refer to the analysis as
 real-time signal extraction.

Real-time signal extraction is considerably more challenging, and useful, than historical signal extraction.
 The difficulty lies in the uncertainty about the future, which is transmitted unto the signal 
 extraction estimates themselves.  One way to conceive of this difficulty is through the warring
 principles of timeliness and accuracy: should we procrastinate in providing our analysis of the present,
 we can increase the accuracy of signal extraction, but our answers become less relevant, even as the present
 time rapidly drifts into the past.  Conversely,   extremely timely extractions suffer from 
 greater future uncertainty, and are likely to exhibit inaccuracy.

There is a considerable body of literature addressing signal extraction, but this book focuses upon
 a particular methodology called Direct Filter Analysis (DFA).  As the original development of DFA
 was univariate, the methodology's power was limited to the information content
 within a single time series.  But because batches of time series can be closely linked, exhibiting 
 correlated trends, common dynamics, or even predictive relationships, it is natural to expect that
 a multivariate extension of DFA to vector time series will more greatly facilitate informed decision 
 making.  The topic of this book is Multivariate Direct Filter Analysis (MDFA).

 

\subsection{The Classic Model-Based Paradigm}

Many signals can be formulated as weighted linear combinations of a time series, in which case the real-time
 signal extraction problem can be approached as a Linear Prediction Problem (LPP).  In order to pose
 an LPP, a solution criterion is needed, and Mean Squared Error (MSE) is often used: one seeks a real-time 
 signal extraction that has minimal MSE discrepancy with the actual target signal.  Although an LPP
 can then be solved, the solution depends on knowing something about the dynamics in the time series process.
 The most venerable approach to understanding these dynamics is to posit a time series model, and fit
 this model to the observed data.  This approach, which goes back to the work of Yule in the 1930s, is called
 the classic paradigm, being based upon a Model-Based Analysis (MBA).

An attractive feature of MBA is that analytical formulas for the LPP solutions can often be obtained, thereby
 facilitating computation.  The philosophy underpinning the classic paradigm is that a Data Generation Process (DGP)
 exists -- as a theoretical, or Platonic construct -- to which the observed data closely adheres.  Formally,
 the DGP is some stochastic process defined upon a probability space, and the observed data is a realization, or sample path, of
 the DGP.  Statistical inference is involved with the science of identifying a model class for the DGP, narrowing down
 the class to a particular model (by eliminating contenders), and fitting that model via fixing values of the parameters.
 Successive applications of model diagnostics allow for refinements, and a process by which we can verify the validity
 of a postulated model.  Of course, all of this is done on the basis of the single realization of the DGP.

While recognizing that any such model need not be correct, i.e., exactly match the DGP itself, such models can yet
 be useful to the extent to which they reflect important features in the data.  Yet it is difficult to keep a model
 simple -- which is necessay to its utility -- and at the same time be sufficiently versatile to explain all the 
data's features.  Moreover, the appellation of importance is subjective: a feature deemed important to one user may
 be irrelevant to another.  This begs the question of customization: each user, with a distinct set of criteria and
 desired applications, could potentially stress the importance of a subset of features at the cost of de-emphasizing others.
 The classic paradigm ignores, or at least passes over, the issue of customization, and proposes a single all-purpose
 concept of utility: the minimization of one-step ahead forecast error MSE.

 Another term for this classic conception of model utility is the Wold decomposition, which breaks a wide class of
 stochastic processes down in terms of a component that is completely predictable from its own infinite past, and 
 a second component fully describable in terms of one-step ahead forecast errors.  Classical models can then
 be viewed as attempts to approximate the linear machinery in the Wold decomposition.    However, were attention to
 focus upon an alternative utility, e.g., 10-step ahead forecasting, a different class of models would be suggested,
 with different apparatus for model selection, fitting, and evaluation.

However, customizing the modeling apparatus to allow for specific applications offers only a partial solution, because
 model mis-specification is the larger challenge.  The full set of LPP solutions for a given time series is greatly
 constrained once a model is introduced, as only a particular subset of solutions can be obtained.  If the model is
 badly mis-specified, the resulting LPP solution will be inadequate, even if the criteria for model selection are customized.
 This empirical disfunctionality motivated the genesis of DFA, which essentially provides access to a much wider
 pool of LPP solutions.  Moreover, the basic DFA can be easily modified to allow for direct customization of 
real-time problems, according to whether users are concerned with timeliness, accuracy, or fidelity to the original signal (called
 smoothness).  
 

%Marc's perspective:
%\begin{itemize}
%\item Maximum Likelihood, main purpose: determine DGP. If DGP is known then optimality can be invoked, in principle. 
%\item Problem: model misspecification. Pseudo maximum likelihood: one-step ahead mean-square criterion. 
%\item Emphasizes short-term performances, only (contrast with real-time trend extraction: long-term component). 
%\item Rigid criterion: can account neither for relevant problem-structure (signal extraction=one and multi-step ahead forecasts) nor for various user-priorities (ATS-trilemma).
%\end{itemize}

\subsection{The Scope of MDFA}

Our critique of the classic paradigm has several facets.  First, there is typically model mis-specification present.  Second, the problem
 has typically not been structured properly, in the sense that the criteria used do not correspond to the relevant LPP, but rather to
   one-step ahead forecasting.  Third, there is no specific customization of the model, in order to account for timeliness and accuracy.
 These weaknesses are actually linked together.  

Model mis-specification is always present; the issue is whether it has a significant impact upon the objectives of analysis.  For instance,
 a given model's mis-specification may have grave repercussions for certain problem structures, while being adequate for other LPPs.
 The given LPP of interest determines the gravity and impact of model mis-specification.  Moreover, in the classic paradigm the one-step
 ahead forecasting LPP is solved, and it is merely hoped that timeliness and accuracy will be adequate for all users.  Model parameters
 can be tweaked, or tuned, in order to indirectly modify timeliness and accuracy -- but the relationships are indirect and often poorly
 understood.  By building the timeliness-accuracy tradeoff directly into the DFA criterion, the optimality of an LPP solution for a
 customized application is assured.

These topics have been treated in Wildi and McElroy (2016) in the case of univariate time series, which discusses at length
 the basic \href{http://blog.zhaw.ch/sef/files/2014/10/DFA.pdf}{DFA} (Sweave environment: replication).  This book presents
 the generalized treatment of the multivariate LPP in   Chapter \ref{mse_sec}. But before discussing customization in Chapter \ref{ats_sec},
 we discuss the applications of forecasting and nowcasting, as well as the impact of data vintage, in  Chapter \ref{fil_sec}).
 Then the basic LPP treatment is extended to nonstationary processes in Chapter \ref{int_sec}, followed by a discussion of filter constraints (Chapter \ref{con_sec}).
  This treatment is extended to the case of co-integration in Chapter \ref{coint_sec}.   Applications to replicating and enhancing classical model-based approaches and HP/CF-filters 
 are given in Chapter \ref{rep_sec}, while  more sophisticated gain/loss structures  are discussed in Chapter \ref{exo_sec}.
Additional topics include inference (Chapter \ref{inf_sec}), regularization (Chapter \ref{reg_sec}), data revisions (Chapter \ref{rev_sec}),
 mixed-frequency data (Chapter \ref{mix_sec}), and   adaptive filtering (Chapter \ref{ada_sec}).

\section{The Style of the Book}

This book was generated using Sweave, in accordance with the philosophy of 
 scientific replicability.  Throughout the text are portions of R code that
 can be pasted into an R script and directly run, given that the user
 has certain packages already installed.  This installation is described below.
 
\subsection{Setting the Paths}

Begin by clearing the workspace: 
\begin{Schunk}
\begin{Sinput}
> #rm(list=ls())
\end{Sinput}
\end{Schunk}
The R code in   various chapters of this book requires installation of the following R packages:
\begin{Schunk}
\begin{Sinput}
> # Load packages: time series and xts
> #library(tseries)
> library(xts)
> # State-space models (will be replicated by MDFA) 
> library(dlm)
> # Classic filter designs (be replicated by MDFA)
> library(mFilter)
> # Numerical package 
> library(numDeriv)
> # Graphical package for recession-shading (empirical examples based on US-GDP)
> library(tis)
> # Library for tables
> library(Hmisc)
> require(xtable)
> #install.packages("devtools")
> library(devtools)
> # Load MDFA package from github
> devtools::install_github("wiaidp/MDFA")
> # MDFA package
> library(MDFA)
\end{Sinput}
\end{Schunk}
US-GDP data for the empirical examples can be retrieved either directly from 
 Quandl (requiring a preliminary user registration) or from a local data folder,
  which is the default-setting:
\begin{Schunk}
\begin{Sinput}
> # Load fresh data from quandl: T/F
> #   Default-setting is False: the data will be loaded from local data folder
> load_from_quandl <- F
\end{Sinput}
\end{Schunk}
Paths to MDFA code, as well as to the US-GDP data, must be provided. 
 It is assumed that the MDFA package is saved to a main folder containing
 subfolders labeled as DFA, MDFA, model-based, and data. 
The R code in the book generates pdf graphs that are saved in a separate folder, 
whose path is specified by {\em path.out}.
\begin{Schunk}
\begin{Sinput}
> # Set main path
> path.main <- paste(getwd(),"/Sweave/",sep="")
> #path.main <- "C:\\Users\\Tucker\\Documents\\MDFAbook\\"
> # Set paths to subfolders
>   # Path to Latex-folder: all pdfs generated by the R code are filed there
> path.out <- paste(path.main,"Latex/",sep="")
>   # Path to data (US-GDP)
> path.dat <- paste(path.main,"Data/",sep="")
>   # Path to code that is part of MDFA-Legacy project but not part of MDFA package 
> path.pgm <- paste(path.main,"R/",sep="")
\end{Sinput}
\end{Schunk}
The univariate DFA code is the same as in \href{http://blog.zhaw.ch/sef/files/2014/10/DFA.pdf}{DFA}; all 
 empirical examples are and will be fully compatible. 

\subsection{DFA}\label{dfa_intro}
We here briefly review the relevant facets of \href{http://blog.zhaw.ch/sef/files/2014/10/DFA.pdf}{DFA},
 thereby providing an anchor for the MDFA discussion. 

\subsubsection{DFT and Periodogram}

The Discrete Fourier Transform (DFT) and the periodogram are defined in Sections 2.2 and 2.3 of
\href{http://blog.zhaw.ch/sef/files/2014/10/DFA.pdf}{DFA}.  
The following periodogram function -- referred to as {\em per} below --
  in the MDFA package replicates these formulae.  Note that frequency $\pi$ is treated differently, depending on
 whether the  sample size is odd or even; also, the value at frequency zero is scaled by $1/\sqrt{2}$,
  which  is explained in later text.  
\begin{Schunk}
\begin{Sinput}
> head(per,100)
\end{Sinput}
\begin{Soutput}
1  function (x, plot_T)                                                              
2  {                                                                                 
3      len <- length(x)                                                              
4      per <- 0:(len/2)                                                              
5      DFT <- per                                                                    
6      for (k in 0:(len/2)) {                                                        
7          cexp <- exp((0+1i) * (1:len) * 2 * pi * k/len)                            
8          DFT[k + 1] <- sum(cexp * x * sqrt(1/(2 * pi * len)))                      
9      }                                                                             
10     if (abs(as.integer(len/2) - len/2) < 0.1)                                     
11         DFT[k + 1] <- DFT[k + 1]/sqrt(2)                                          
12     per <- abs(DFT)^2                                                             
13     if (plot_T) {                                                                 
14         par(mfrow = c(2, 1))                                                      
15         plot(per, type = "l", axes = F, xlab = "Frequency", ylab = "Periodogram", 
16             main = "Periodogram")                                                 
17         axis(1, at = 1 + 0:6 * len/12, labels = c("0", "pi/6",                    
18             "2pi/6", "3pi/6", "4pi/6", "5pi/6", "pi"))                            
19         axis(2)                                                                   
20         box()                                                                     
21         plot(log(per), type = "l", axes = F, xlab = "Frequency",                  
22             ylab = "Log-periodogram", main = "Log-periodogram")                   
23         axis(1, at = 1 + 0:6 * len/12, labels = c("0", "pi/6",                    
24             "2pi/6", "3pi/6", "4pi/6", "5pi/6", "pi"))                            
25         axis(2)                                                                   
26         box()                                                                     
27     }                                                                             
28     return(list(DFT = DFT, per = per))                                            
29 }                                                                                 
\end{Soutput}
\end{Schunk}
This function will be generalized in the new multivariate setting.

\subsubsection{Basic DFA}

A simple   version of the DFA  based on the MSE criterion alone -- 
 as proposed in Section 4.1 of \href{http://blog.zhaw.ch/sef/files/2014/10/DFA.pdf}{DFA} --
 is included in the MDFA package:  

\begin{Schunk}
\begin{Sinput}
> # This function computes MSE DFA solutions 
> # L is the length of the MA filter,
> # periodogram is the frequency weighting function in the DFA
> # Gamma is the transfer function of the symmetric filter (target) and
> # Lag is the lag-parameter: Lag=0 implies real-time filtering, Lag=L/2
> #     implies symmetric filter
> # The function returns optimal coefficients as well as the transfer 
> #     function of the optimized real-time filter
> head(dfa_ms,100)
\end{Sinput}
\begin{Soutput}
1  function (L, periodogram, Lag, Gamma)                                    
2  {                                                                        
3      periodogram[1] <- periodogram[1]/2                                   
4      K <- length(periodogram) - 1                                         
5      X <- exp(-(0+1i) * Lag * pi * (0:(K))/(K)) * rep(1, K + 1) *         
6          sqrt(periodogram)                                                
7      X_y <- exp(-(0+1i) * Lag * pi * (0:(K))/(K)) * rep(1, K +            
8          1)                                                               
9      for (l in 2:L) {                                                     
10         X <- cbind(X, (cos((l - 1 - Lag) * pi * (0:(K))/(K)) +           
11             (0+1i) * sin((l - 1 - Lag) * pi * (0:(K))/(K))) *            
12             sqrt(periodogram))                                           
13         X_y <- cbind(X_y, (cos((l - 1 - Lag) * pi * (0:(K))/(K)) +       
14             (0+1i) * sin((l - 1 - Lag) * pi * (0:(K))/(K))))             
15     }                                                                    
16     xtx <- t(Re(X)) %*% Re(X) + t(Im(X)) %*% Im(X)                       
17     b <- as.vector(solve(xtx) %*% (t(Re(X_y)) %*% (Gamma * periodogram)))
18     trffkt <- 1:(K + 1)                                                  
19     trffkt[1] <- sum(b)                                                  
20     for (k in 1:(K)) {                                                   
21         trffkt[k + 1] <- (b %*% exp((0+1i) * k * (0:(length(b) -         
22             1)) * pi/(K)))                                               
23     }                                                                    
24     return(list(b = b, trffkt = trffkt))                                 
25 }                                                                        
\end{Soutput}
\end{Schunk}
This function is nested in the multivariate MDFA,
  in the sense that the latter can replicate the former perfectly when suitably parametrized;
 see Section \ref{ex_rep_dfa} below.



\subsubsection{Customized DFA}

A more general DFA function, called \emph{dfa\textunderscore analytic}, is proposed in Section 4.3.5 of
\href{http://blog.zhaw.ch/sef/files/2014/10/DFA.pdf}{DFA}. Customization and the generic 
 Accuracy-Timeliness-Smoothness (ATS) trilemma are presented in Sections 4.3 and 5 of
 \href{http://blog.zhaw.ch/sef/files/2014/10/DFA.pdf}{DFA}. This function is included in the MDFA package: 
\begin{Schunk}
\begin{Sinput}
> head(dfa_analytic)
\end{Sinput}
\begin{Soutput}
1 function (L, lambda, periodogram, Lag, Gamma, eta, cutoff, i1, 
2     i2)                                                        
3 {                                                              
4     periodogram[1] <- periodogram[1]/2                         
5     lambda <- abs(lambda)                                      
6     eta <- abs(eta)                                            
\end{Soutput}
\end{Schunk}
The additional control parameters {\em lambda}, {\em eta} allow for customization of the filter,  as discussed below
 in Chapter \ref{ats_sec}.  The Boolean {\em i1} and {\em i2}
  can enforce useful filter constraints; see Chapter \ref{con_sec}. This function is also encompassed by the   MDFA. 



\subsection{MDFA}\label{mdfa_intro}

The R code for MDFA is more sophisticated than that of the DFA, and is correspondingly more complex and lengthy. 
 As for the DFA package, the MDFA code can be sourced. We here briefly review the corresponding pieces.


\subsubsection{Data Matrix}

All time series are collected in a data-\emph{matrix}, say $X$, which is organized as follows: 
\begin{itemize}
\item the first column $X[,1]$ of $X$ always corresponds to the target series: the target series $X[,1]$ is the time series
 to be forecasted, nowcasted or backcasted.
\item Columns $2$, $3$, $\ldots$ of $X$ are allocated to the explanatory variables (more than one in a multivariate setting). 
If the target series is part of the set of explanatory variables (it does not have to be), then it must be assigned a specific column 
-- by convention always the second one -- in $X$, i.e., in this case the target series is entered twice, in the first column (target) and
  in the second column (explanatory data).     
\end{itemize}

\noindent {\bf Example}.  Suppose we study a  two-dimensional signal extraction problem, whereby the target series (first column) 
is part of the set of explanatory variables:
\begin{Schunk}
\begin{Sinput}
> set.seed(1)
> len <- 100
> target <- arima.sim(list(ar=0.9),n=len)
> explanatory_2 <- target+rnorm(len)
> explanatory <- cbind(target,explanatory_2)
> x <- cbind(target,explanatory)
> dimnames(x)[[2]] <- c("target","explanatory 1","explanatory 2")
> head(x)
\end{Sinput}
\begin{Soutput}
       target explanatory 1 explanatory 2
[1,] 1.703613      1.703613     0.3191863
[2,] 1.398197      1.398197     3.2674879
[3,] 3.659995      3.659995     4.0850957
[4,] 3.254756      3.254756     3.0161086
[5,] 3.619020      3.619020     4.6775026
[6,] 3.285120      3.285120     4.1715424
\end{Soutput}
\end{Schunk}
For a one-step ahead forecast LPP, we might consider lagging both the explanatory variables:
\begin{Schunk}
\begin{Sinput}
> x<-cbind(x[,1],lag(x[,2:3],-1))
> dimnames(x)[[2]]<-c("target","lagged explanatory 1","lagged explanatory 2")
> head(x)
\end{Sinput}
\begin{Soutput}
       target lagged explanatory 1 lagged explanatory 2
[1,] 1.703613                   NA                   NA
[2,] 1.398197             1.703613            0.3191863
[3,] 3.659995             1.398197            3.2674879
[4,] 3.254756             3.659995            4.0850957
[5,] 3.619020             3.254756            3.0161086
[6,] 3.285120             3.619020            4.6775026
\end{Soutput}
\end{Schunk}
 By adopting the frequency-domain methods of this book, we can generalize this construction and
    avoid the introduction of missing values (denoted by NA in R).  $\quad \Box$


\subsubsection{DFT}

In contrast to the univariate DFA, where the LPP can be expressed in terms of  the periodogram, the multivariate case 
 requires the   DFT of each time series in order to account for cross-sectional dependencies.  These DFTs are complex-valued
 quantities, and the angular portion of the cross-spectrum provides information about the relative phase-shift of each explanatory time series. 
  In the univariate case the relative phase-shift is irrelevant, because the target series and the explanatory series are identical.
 The scope of the method is extended in order to cover the mixed-frequency case, which is discussed in Chapter \ref{mix_sec}. 
 Another facet, is that we allow for the possibility of integrated processes; see Chapter \ref{int_sec}. 
 In order to illustrate some of the new features we briefly look at the main DFT function called {\em spec\textunderscore comp}:
\begin{Schunk}
\begin{Sinput}
> spec_comp
\end{Sinput}
\begin{Soutput}
function (insamp, x, d) 
{
    if (d == 1) {
        weight_func <- periodogram_bp(diff(x[1:insamp, 1]), 1, 
            insamp - 1)$fourtrans
        if (length(weight_func) > 1) {
            for (j in 2:ncol(x)) {
                per <- periodogram_bp(diff(x[1:insamp, j]), 1, 
                  insamp - 1)$fourtrans
                weight_func <- cbind(weight_func, per)
            }
        }
    }
    else {
        weight_func <- periodogram_bp(x[1:insamp, 1], 0, insamp)$fourtrans
        if (length(weight_func) > 1) {
            for (j in 2:ncol(x)) {
                per <- periodogram_bp(x[1:insamp, j], 0, insamp)$fourtrans
                weight_func <- cbind(weight_func, per)
            }
        }
    }
    colnames(weight_func) <- colnames(x)
    return(list(weight_func = weight_func))
}
<bytecode: 0x000001ea32507de8>
<environment: namespace:MDFA>
\end{Soutput}
\end{Schunk}
The inner loop   tracks the columns of the data matrix $X$ and the DFTs are stored in a matrix called \emph{weight\textunderscore func},
  which is returned by the function. The matrix \emph{weight\textunderscore func} collects all DFTs;
  the target series is always in the first column, whereas the DFTs of the explanatory series are in columns $2$, $3$, $\ldots$
 The function \emph{periodogram\textunderscore bp}, called in the above loop, is slightly more general than the DFA 
function \emph{per} proposed in the previous section. In particular, it can handle various integration orders as well as
 seasonal peculiarities. 

\subsection{Using MDFA}\label{control_dfa}

\subsubsection{A Versatile User Interface}

MDFA is a generic forecast and signal extraction paradigm. Besides its capacity to  replicate classical time series approaches, 
  MDFA possesses unique features such as customization and regularization (Chapter \ref{reg_sec}); it can
  treat data revisions (Chapter \ref{rev_sec}), mixed-frequency problems (Chapter \ref{mix_sec}),
 and non-stationarity (Chapters \ref{int_sec} and \ref{coint_sec}. Accordingly, the user interface 
is more sophisticated than the precediing DFA package.
Consider the head of the main estimation routine:    

\begin{Schunk}
\begin{Sinput}
> head(mdfa_analytic)
\end{Sinput}
\begin{Soutput}
1 function (L, lambda, weight_func, Lag, Gamma, eta, cutoff, i1,             
2     i2, weight_constraint, lambda_cross, lambda_decay, lambda_smooth,      
3     lin_eta, shift_constraint, grand_mean, b0_H0, c_eta, weight_structure, 
4     white_noise, synchronicity, lag_mat, troikaner)                        
5 {                                                                          
6     lambda <- abs(lambda)                                                  
\end{Soutput}
\end{Schunk}
 Arguments such as  \emph{weight\textunderscore func} (discussed above), the filter length ($L$), and the target specification \emph{Gamma}
 are straightforward.    But there are numerous additional control parameters: the relevance and the modus operandi of these
 will be discussed in this book. 


\subsubsection{Default Settings}

For convenience, we store a so-called default setting of the parameters in a file called \emph{control\textunderscore default}.
 First we define the data (initialize the DFT matrix) and specify the filter  length:
\begin{Schunk}
\begin{Sinput}
> weight_func <- matrix(rep(1:6,2),ncol=2)
> L <- 2
\end{Sinput}
\end{Schunk}
Given these two entries (DFT and filter length), the default-settings are as follows:
\begin{Schunk}
\begin{Sinput}
> d<-0
> lin_eta<-F
> lambda<-0
> Lag<-0
> eta<-0
> i1<-F
> i2<-F
> weight_constraint<-rep(1/(ncol(weight_func)-1),ncol(weight_func)-1)
> lambda_cross<-lambda_smooth<-0
> lambda_decay<-c(0,0)
> lin_expweight<-F
> shift_constraint<-rep(0,ncol(weight_func)-1)
> grand_mean<-F
> b0_H0<-NULL
> c_eta<-F
> weights_only<-F
> weight_structure<-c(0,0)
> white_noise<-F
> synchronicity<-F
> cutoff<-pi
> lag_mat<-matrix(rep(0:(L-1),ncol(weight_func)),nrow=L)
> troikaner<-F
\end{Sinput}
\end{Schunk}
This particular configuration will be used extensively in Chapter \ref{mse_sec}; it corresponds to the basic MSE criterion 
 (i.e., no customization) without regularization, without design constraints, and without any {\em a priori} knowledge.
  Also, this configuration presumes a   common identical sampling frequency (i.e., no mixed frequency data)
 and the absence of data revisions. The default settings can be obtained by sourcing the corresponding R file:

\begin{Schunk}
\begin{Sinput}
> source(file=paste(path.pgm,"control_default.r",sep=""))
\end{Sinput}
\end{Schunk}
For later use we   source a convenient plotting function:
\begin{Schunk}
\begin{Sinput}
> source(file=paste(path.pgm,"mplot_func.r",sep=""))
\end{Sinput}
\end{Schunk}

\subsubsection{Selected Calls: Classic MSE, Customization and Regularization}

Selected calls of the classic MSE  criterion -- as well as calls utilizing the customization or regularization features --
 are available through dedicated functions in the MDFA package: 
\begin{Schunk}
\begin{Sinput}
> head(MDFA_mse)
\end{Sinput}
\begin{Soutput}
1 function (L, weight_func, Lag, Gamma) 
2 {                                     
3     cutoff <- pi                      
4     lin_eta <- F                      
5     lambda <- 0                       
6     eta <- 0                          
\end{Soutput}
\begin{Sinput}
> head(MDFA_mse_constraint)
\end{Sinput}
\begin{Soutput}
1 function (L, weight_func, Lag, Gamma, i1, i2, weight_constraint, 
2     shift_constraint)                                            
3 {                                                                
4     cutoff <- pi                                                 
5     lin_eta <- F                                                 
6     lambda <- 0                                                  
\end{Soutput}
\begin{Sinput}
> head(MDFA_cust)
\end{Sinput}
\begin{Soutput}
1 function (L, weight_func, Lag, Gamma, cutoff, lambda, eta)                  
2 {                                                                           
3     lin_eta <- F                                                            
4     weight_constraint <- rep(1/(ncol(weight_func) - 1), ncol(weight_func) - 
5         1)                                                                  
6     lambda_cross <- lambda_smooth <- 0                                      
\end{Soutput}
\begin{Sinput}
> head(MDFA_cust_constraint)
\end{Sinput}
\begin{Soutput}
1 function (L, weight_func, Lag, Gamma, cutoff, lambda, eta, i1, 
2     i2, weight_constraint, shift_constraint)                   
3 {                                                              
4     lin_eta <- F                                               
5     lambda_cross <- lambda_smooth <- 0                         
6     lambda_decay <- c(0, 0)                                    
\end{Soutput}
\begin{Sinput}
> head(MDFA_reg)
\end{Sinput}
\begin{Soutput}
1 function (L, weight_func, Lag, Gamma, cutoff, lambda, eta, lambda_cross,    
2     lambda_decay, lambda_smooth, troikaner = F, b0_H0 = NULL)               
3 {                                                                           
4     lin_eta <- F                                                            
5     weight_constraint <- rep(1/(ncol(weight_func) - 1), ncol(weight_func) - 
6         1)                                                                  
\end{Soutput}
\begin{Sinput}
> head(MDFA_reg_constraint)
\end{Sinput}
\begin{Soutput}
1 function (L, weight_func, Lag, Gamma, cutoff, lambda, eta, lambda_cross,      
2     lambda_decay, lambda_smooth, i1, i2, weight_constraint, shift_constraint, 
3     troikaner = F, b0_H0 = NULL)                                              
4 {                                                                             
5     lin_eta <- F                                                              
6     lin_expweight <- F                                                        
\end{Soutput}
\end{Schunk}
The heads of the corresponding functions differ in the number of additional arguments available 
when going from specific (MSE) to generic (reg).  The following chapters of the book provide an understanding of
 the use of these functions.



%----------------------------------------

% Chapter 2


\chapter{Linear Prediction Problems}
\label{chap:lpp}

\section{Background on Stationary Vector Time Series}

The reader should have a basic familiarity with multivariate time
 series analysis, such as that provided by L\"utkepohl (2007).  
 Our focus is on discrete-time stochastic processes taking values in $\RR^n$,
 and such will be denoted $\{ X_t \}$, i.e., a vector time series.
   Each $X_t$ for  a particular   $t \in \ZZ$ is a random vector 
 with $n$ components, and 
 the $j$th component will be denoted $X_{t,j}$ for $1 \leq j \leq n$.
  This can also be written as $X_{t,j} = e_j^{\prime} \, X_t$, where
 $e_j$ is the $j$th unit vector in $\RR^n$.  The union of these
 unit vectors is the $n \times n$ identity matrix, denoted $1_n$.

 In this book we are focused upon square integrable random variables,
 so that the classic Hilbert space projection theory (see, for example,
 Brockwell and Davis (1991)) can be applied.   Occasionally, we consider 
  vector time series $\{ Y_t \}$ or $\{ Z_t \}$, in which case the 
 same conventions apply.When $\{ X_t \}$ is 
 weakly stationary, its autocovariance function (acf)
 is defined for $h \in \ZZ$  via 
\[
   \Gamma (h) = \mbox{Cov} [ X_{t+h}, X_t ],
\]
 which does not depend upon $t$ by the stationarity assumption.  Recall
 that $\Gamma (-h) = { \Gamma (h) }^{\prime}$, and clearly
\[
   \Gamma_{jk} (h) = \mbox{Cov} [ X_{t+h,j}, X_{t,k} ]
\]
 for $1 \leq j,k \leq n$. The spectral density is a complex matrix-valued
 function of $\omega \in [-\pi, \pi]$, defined as the 
 Fourier Transform (FT) of the acf:
\[
   F (\omega) = \sum_{h \in \ZZ} \Gamma (h) \, z^h,
\]
 where we use the shorthand $z = e^{-i \omega}$.  Clearly,
\[
    F(-\omega) = 
   \sum_{h \in \ZZ} \Gamma (h) \, z^{-h} 
   = \sum_{h \in \ZZ} \Gamma (-h) \, z^h = 
  \sum_{h \in \ZZ} { \Gamma (h) }^{\prime} \, z^h = { F (\omega) }^{\prime},
\]
 which shows that the spectral density function (sdf) is Hermitian.
  In addition, its eigenvalues (for each $\omega$) are real and non-negative.
  Given a bounded sdf (i.e., each $F_{jk} $ has bounded modulus as a function
 of $\omega$), the acf can be recovered via inverse FT:
\begin{equation}
 \label{eq:spec2acf}
  \Gamma (h) = { \langle F  \rangle }_h
  =  \frac{1}{2 \pi} \int_{-\pi}^{\pi} F(\omega) \, e^{i \omega h}
  \, d\omega,
\end{equation}
 which uses the bracket notation to define the average integral of a 
 function (of $\omega$) multiplied by $e^{i \omega h} = z^{-h}$.

 The lag operator on a time series is denoted $L$, and is defined via
 the action
\[
  L X_t = X_{t-1}.
\]
  Powers  of $L$ are defined analogously, with $L^0 = 1$ (an operator
 identity) and negative powers yielding forward time shifts, i.e., leads.
 Matrix polynomials of $L$ yield new operators that act upon a time series
 using the linearity principal.  Thus, if $A(L) = \sum_{j=0}^a A_j \, L^j$
 for $n \times n$ matrices $A_j$, then
\[
  A(L) \, X_t = \sum_{j=0}^a A_j \, X_{t-j}.
\] 
  For many applications in this book, the polynomials are actually scalar,
 and can be interpreted as having coefficients $A_j$ given by an  
 identity matrix $1_n$ multiplied by a scalar coefficient $a_j$.
 
 The spectral representation of a stationary square integrable vector 
time series is particularly useful.  Assuming that $\EE X_t = 0$ (the
 zero vector in $\RR^n$) so that no fixed effects are present, we describe
 the stochastic process via
\begin{equation}
\label{eq:specRep}
  X_t = \int_{-\pi}^{\pi} e^{i \omega t} \, \mathcal{Z} (d\omega),
\end{equation}
 which is a stochastic integral computed with a Stieltjes measure
 $\mathcal{Z} (d\omega)$.  This is an orthogonal increments process,
 which mean $\mathcal{Z}$ is a random measure defined on ${[-\pi,\pi]}^n$
 that maps disjoint sets to independent random variables.  The actual
 distribution of the random measure is not our concern, but the particular
 orthogonal increments process associated with $\{ X_t \}$ has the 
 property that
\[
  \mbox{Cov} [ \mathcal{Z} (d\omega), \mathcal{Z} (d\xi) ]
   = {(2 \pi)}^{-1} \, F (\omega) \, d\omega \, 1_{ \{   \omega = \xi \} }
\]
 where $1_A$ is the indicator for the set $A$.
  (Also recall that for complex variables, a 
covariance involves conjugation of
 the second argument.)  This validates the expression
\[
  \mbox{Cov} [ X_{t+h}, X_t ] =
 \int_{-\pi}^{\pi}  \int_{-\pi}^{\pi} e^{i \omega (t+h)} \,
  e^{-i \xi t } \, \mbox{Cov} [ \mathcal{Z} (d\omega), 
  \mathcal{Z} (d\xi) ] 
 =  \frac{1}{2 \pi} \int_{-\pi}^{\pi} F(\omega) \, e^{i \omega h}
  \, d\omega = \Gamma (h).
\]
 As an example -- that furnishes a basic building block for subsequent processes --
 we have {\em white noise}, which refers to a mean zero $\{ X_t \}$ where 
 $F$ is constant, i.e.,
 $F(\omega) = \Sigma$ for all $\omega$, where $\Sigma $ is real and symmetric and
 non-negative definite.  Clearly, $\Gamma (h) = 0$ for $h \neq 0$ and $\Gamma (0) = \Sigma$.
 We denote this type of process with the notation $\mbox{WN} (\Sigma)$.

 The advantage of the spectral representation is that it quickly facillitates
 the understanding of linear filtering in the frequency domain.
 A multivariate filter maps one vector time series to another, and for
 now we suppose that both input and output belong to $\RR^n$.
 Linear filters can be expressed as matrix Laurent series in $L$:
\[
 \Psi (L) = \sum_{\ell \in \ZZ} \psi (\ell) \, L^{\ell},
\]
 where each $\psi (\ell)$ is an $n \times n$ matrix.  
 Individual entries of the matrix are denoted $\psi_{jk} (\ell)$,
 for $1 \leq j, k \leq n$.   We also use the notation
  $ {[\Psi (L) ]}^{s}_r $ to denote $\sum_{\ell=r}^s
  \psi (\ell) \, L^{\ell}$. 
     Another quantity of interest is the derivative of a filter, 
  defined via $\partial \Psi (L) = \sum_{j \in \ZZ} j \, \psi(j) \, L^{j-1}$.

  
  In the special case that $\Psi (L)$ is a power
 series in $L$, the only nonzero coefficients are for $\ell \geq 0$, 
 so that no negative powers of $L$ are featured, i.e., the filter
 only utilizes present and past data.  Such a filter is called a 
 {\em concurrent} filter.  
  The action of   a linear  filter
 on a weakly stationary time series,
 expressed in terms of the spectral representation, is
\[
  Y_t = \Psi (L) \, X_t = \sum_{\ell \in \ZZ} \psi (\ell) \, X_{t-\ell}
   = \sum_{\ell \in \ZZ} \psi (\ell)
  \, \int_{-\pi}^{\pi} e^{i \omega (t-\ell)} \,
   \mathcal{Z} (d\omega) =
  \int_{-\pi}^{\pi} e^{i \omega t} \, \Psi (e^{-i \omega}) \,
   \mathcal{Z} (d\omega).
\]
 So the output time series $\{ Y_t \}$ has orthogonal increments process
  $\Psi (z) \, \mathcal{Z} (d\omega)$, and in particular its sdf is
\[  
   \Psi (z) \, F(\omega) \, { \Psi (z) }^{*},
\]
 where $*$ denotes the conjugate transpose.  Thus, it is very natural
 to analyze a filter in terms of the function $\Psi (e^{-i \omega})$,
 which is called the {\em frequency response function} (frf).  In 
 the scalar case, an frf can be further dissected via the polar 
 decomposition of a complex number, yielding its {\em gain function}
 (the modulus) and the {\em phase function} (its angular portion).  
  Note that the coefficients are recovered from the frf via the inverse FT:
\[
  \psi (\ell) = {\langle \Psi (e^{-i \cdot } ) \rangle }_{\ell}.
\]
  It is well-known from Fourier theory that the degree of smoothness of a
 function at $\omega = 0$ corresponds to the degree of decay in the coefficients 
 of its  inverse FT.  In particular, when the frf is smooth and flat in a
 neighborhood of the origin then the matrix norm of the 
 coefficients $\psi (\ell)$ decays rapidly as $|\ell| \tends \infty$.  Conversely,
 discontinuities in the frf indicates slowly decaying coefficients.

 Datasets are typically available as a finite set of contiguous regular
 measurements, denoted $\{ x_1, x_2, \ldots, x_T \}$, where $T$ is 
 the length of sample.  The data is viewed as a realization of the 
 corresponding random vectors $\{ X_1, X_2, \ldots, X_T \}$, or alternatively
 as a time window of the sample path $\{ x_t \}$ corresponding to times
 $1, 2, \ldots, T$.  Applying the {\em vec} operator to such a sample
 yields the full vector $\underline{X}$, which is given by
\[
 \underline{X} = \mbox{vec} [ X_1, X_2, \ldots, X_T ].
\]
 The covariance matrix of this $nT$-dimensional random vector, in the
 stationary case, is block Toeplitz,   Each block is $n \times n$, and
 the $st$th such block, for $1 \leq s,t \leq T$, is given by $\Gamma (s-t)$.
 Also, from the sample we can compute the {\em Discrete Fourier Transform} (DFT)
  via
\begin{equation}
\label{eq:dft-def}
    \widetilde{X} (\omega) = T^{-1/2} \, \sum_{t=1}^T z^t \, X_t.
\end{equation}
 This can be computed for any $\omega \in [-\pi, \pi]$, though if we restrict
 to Fourier frequencies -- of the form
 $2 \pi j/T$ for integer $j$ -- then the various real and imaginary 
components of the DFT will be asymptotically uncorrelated, and
 also asymptotically normal.   The multivariate {\em periodogram}
  is defined to   be the rank one Hermitian matrix
\begin{equation}
\label{eq:per-def}
  \widehat{F} (\omega) = \widetilde{X} (\omega) \, { \widetilde{X} (\omega) }^*.
\end{equation}
  The periodogram furnishes a basic estimate of the spectral density $F$ of the process.
  There is an empirical version of (\ref{eq:spec2acf}), where the periodogram is mapped to the sample 
 autocovariance:
\begin{equation}
 \label{eq:per2acf}
  \widehat{\Gamma} (h) = { \langle \widehat{F}  \rangle }_h
  =  \frac{1}{2 \pi} \int_{-\pi}^{\pi} \widehat{F}(\omega) \, e^{i \omega h}
  \, d\omega.
\end{equation}
   This is easily verified using the definition of sample autocovariance
\[
   \widehat{\Gamma} (h) = T^{-1} \, \sum_{t=1}^{T-h}
   X_{t+h} \, X_t^{\prime}
\]
 for $h \geq 0$, and with $\widehat{\Gamma} (h) = { \widehat{\Gamma} (-h) }^{\prime}$ for $h < 0$.
  Conversely, the periodogram is the FT of the sample autocovariances:
\begin{equation}
 \label{eq:acf2per}
   \widehat{F} (\omega) = \sum_{|h| < T} \widehat{\Gamma} (h) \, e^{-i   \omega h}.
\end{equation}

\section{MSE Optimal Prediction Problems}

\subsection{The Linear Prediction Problem} We define the class of
real-time estimation problems considered in this book.  This chapter
 focuses upon the case of weakly stationary vector time series, but
 Chapter \ref{chap:int} makes extensions to difference stationary processes.

\begin{Definition} \rm
\label{def:target}
 A {\bf target} is defined to be the output of any known linear
 filter acting on the data process, i.e.,  $\{Y_t \}$ is a target
 time series corresponding to a given filter $\Psi (L)$ acting on a 
given observed time series
 $\{ X_t \}$ if and only if we can write for all integers $t$
\[
 Y_t = \Psi (L) X_t.
\] 
 We say that $\{ Y_t \}$ is a {\bf scalar target} if $\Psi (L)$ is a
 $1 \times n$-dimensional filter.
\end{Definition}

We are only interested in scalar targets.  The reason is that if $\{ Y_t \}$
 is multivariate, we can treat each component series $\{ Y_{t,j} \}$ for
 $1 \leq j \leq n$ in turn, so that without loss of generality we
 can just give the treatment for the scalar case. 

\begin{Example} {\bf  Multi-step Ahead Forecasting.}   \rm
\label{exam:multi-step.fore}
  Suppose that our goal is to forecast one of the component series 
 $h$ steps ahead, where $h \geq 1$ is the given {\em forecast lead}.
  Here, suppose that the series of interest is the first component, so 
 that 
\[
  Y_t = X_{t+h,1}
\]
  for all $ t \in \ZZ$.  This is indeed a scalar target, setting
  $\Psi (L) = L^{-h} \, e_1^{\prime}$.  That is, each $\psi (\ell)$
 is a $1 \times n$ row vector, each of which are zero except $\psi (-h)$,
 which is given by $e_1^{\prime}$.
\end{Example}

\begin{Example} {\bf Ideal Low-Pass.} \rm
\label{exam:ideal-low}
 In order to estimate a trend from a given series, conceptually we wish
 to screen out all the higher frequency components in the data.  With reference
 to the spectral representation, if $\Psi (z)$ is zero for all $\omega$ in a 
 band of the higher frequencies, then $\{ Y_t \}$ will only be composed of 
 low frequency stochastic sinusoids.  The simplest way to achieve such an output
 is to design the frf as an indicator function, involving 
  a steep cutoff of noise frequencies; see Baxter and King (1999).  This is
 viewed by some as the best possible definition of trend, and hence the filter is
 called the ideal low-pass.  For scalar target, we have
\[
  \Psi (z) = 1_{ [ -\mu, \mu ]} (\omega) \, e_1^{\prime}
\]
 for some cutoff $\mu \in (0, \pi)$ that separates the pass-band from
the stop-band.  To understand this terminology of pass-band and stop-band, observe
 that the spectral representation of the scalar target is
\[
  Y_t = \int_{ [-\mu, \mu]} e^{i \omega t} \, e_1^{\prime} 
  \mathcal{Z} (d\omega).
\]
  Here, the stochastic integration only includes frequencies in the pass-band 
 $[-\mu, \mu]$,  and all content belonging to the stop-band has been eliminated.
 The coefficients are given by 
\[ 
  \psi (\ell) = \frac{ \sin (\ell \mu) }{ \pi \ell } \, e_1^{\prime}
\]
 for $\ell \neq 0$ and $\psi (0) = \mu/\pi \, e_1^{\prime}$.   
\end{Example}

\begin{Example} {\bf Ideal Band-Pass.} \rm
\label{exam:ideal-bp}
  To extract  the business cycle, or
  some other mid-range dynamics present in the time series,
  we can screen out all lower and higher frequencies while preserving
   the given band.  This can be accomplished by adapting the
   low-pass filter: we seek a filter $\Psi (z)$  that is zero unless 
   $|\omega| \in  [\mu, \eta]$ with $0 < \mu < \eta$,
   which can be constructed by taking the difference of two low-pass
   filters with cutoffs $\mu$ and $\eta$.  For a scalar target, we have
  \[
  \Psi (z) = \left( 1_{ [ -\eta, \eta ]} (\omega) - 
    1_{ [ -\mu, \mu ]} (\omega) \right) \, e_1^{\prime}.
\]
Then $\{ Y_t \}$ will only be composed of frequencies 
between $\mu$ and $\eta$.  This is called the ideal band-pass filter,
 and has been advocated for use in econometrics by Christiano and Fitzgerald (2003).
\end{Example}

\begin{Example} {\bf HP Low-pass.} \rm
\label{exam:hp-low}
 The Hodrick-Prescott (HP) filter (Hodrick
and Prescott, 1997) is a low-pass filter appropriate for producing trends.
 A multivariate version of the HP low-pass (or just HP), 
 associated with trend-irregular structural models,
 was proposed in McElroy and Trimbur (2015); the frf is given by
\[
   \Psi (z) = \Sigma_{\mu} \, { \left( \Sigma_{\mu} + {|1 - z|}^4 \, \Sigma_{\iota} 
	\right) }^{-1}
\]
 in the case that the matrices $\Sigma_{\mu}$ and $\Sigma_{\iota}$ have full rank.  
  When $\Sigma_{\mu}$ has reduced rank, an alternative expression is available,
 but note that the frf is a continuous matrix-valued function of $\omega$.
The matrices $\Sigma_{\mu}$ and $\Sigma_{\iota}$ have an interpretation in terms
 of the econometric concept of trend co-integration, which is further explored
 in Chapter \ref{chap:coint}.  It is always assumed that $\Sigma_{\iota}$ has full rank, 
 and hence  we can rewrite as
\[
  \Psi (z) = Q \, { \left( Q + {| 1 - z|}^4 \, 1_n \right) }^{-1}
\]
 with $Q = \Sigma_{\mu} \, \Sigma_{\iota}^{-1}$ representing a matrix
 {\em signal-to-noise ratio}, or snr.  This formula 
  generalizes the univariate HP filter, which has frf
\[
 \Psi (z) = \frac{ q}{ q + {| 1 - z|}^4 }
\]
 for snr parameter $q > 0$.  Small values of $q$ correspond to trends that are
 buried in volatile white noise, and thus require much smoothing to recover.
 The filter perfectly reflects this need, because a small $q$ indicates a steep
 drop in the frf (which takes value one at $\omega = 0$) as $\omega$ is increased
 from zero, and hence the filter coefficients decay slowly.  Conversely, higher values
 of $q$ -- corresponding to highly salient trends -- yield an frf that equals unity
 in a large neighborhood of the origin, with coefficients that decay swiftly,
 indicating that little smoothing is needed to discover the trend.  These observations
 carry over to the multivariate case, though we judge the size of the snr via a
 matrix norm (such as the maximum eigenvalue) of $Q$.  Some of these eigenvalues can be
 zero, corresponding to the case that $\Sigma_{\mu}$ has reduced rank -- this has the
 effect of generating trends that are collinear.  In the case of a scalar target,
 where we seek a trend for the first input series $\{ X_{t,1} \}$,  we have 
\begin{equation}
\label{eq:hp.mvar-def}
  \Psi (z) = e_1^{\prime} \, Q \, { \left( Q + {| 1 - z|}^4 \, 1_n \right) }^{-1}.
\end{equation}
 There are no known analytical formulas for the coefficients in the multivariate case,
 although in the univariate case they are available in McElroy (2008b).
\end{Example}

\begin{Example} {\bf HP High-pass.} \rm
\label{exam:hp-high}
 While the HP filter is used to extract
 trends, the residual is thought to measure the business cycle along with
 higher frequency oscillations in the data.  Thus, taking the identity minus the
 HP low-pass yields the HP high-pass filter:
\[
  \Psi (z) =   { \left( Q + {| 1 - z|}^4 \, 1_n \right) }^{-1} \, {|1 - z|}^4.
\]
  The presence of the term ${|1 - z|}^4$ indicates differencing by the ${(1-L)}^2$
 and ${(1 - L^{-1})}^2$; thus the HP high-pass will annihilate cubic polynomials,
 and generally reduces high order stochastic trends to stationarity.  
\end{Example}

%\vspace{.5cm}

As we see from these examples, the targets of real-time signal 
 extraction  are features of the stochastic process that are of interest to 
 a particular user.  Some scalar targets depend upon only  a single component 
 of the time series (Examples \ref{exam:multi-step.fore} and \ref{exam:ideal-low}), 
whereas others may be defined in
 terms of all the components (Examples \ref{exam:hp-low} and \ref{exam:hp-high}). 
 However, these targets
 represent an ideal feature of the time series that typically we cannot compute
 in real-time.

 Real-time refers to time present, wherein we have access to present and past 
 information, but have great uncertainty about the future.  This is an 
 essential feature of human existence.  Time series methodology provides tools
 to model and understand the flow of information from past to present to future,
 with the implicit viewpoint that whereas causality is to some degree present --
 past events have  a causative impact on future events, but not vice versa --
 there are other facets governing present and future outcomes that are not 
 traceable to a particular variable's past.  In other words, knowing the past
 values of a component time series $\{ X_{t,1} \}$ is not sufficient to flawlessly
 determine its future values.  However, having other explanatory variables in play
 can reduce the uncertainty of the future; taking $n$ higher, we may be able
 to reduce the errors in forecasts.  

 The concept of {\em Granger causality} can be used to parse these notions mathematically.
 We may consider other component series $\{ X_{t,j} \}$ for $j \geq 2$ useful
 for determining the future of $\{ X_{t,1} \}$ if the one-step ahead forecast
 mean square error (MSE) is reduced, in which case we say that Granger causality
 is present.  In such a scenario it can be proved that the one-step ahead forecast
 MSE arising from utilizing $\{ X_{t,1} \}$ alone is greater than that obtained
 using the additional series.  Hence, there is benefit to increasing $n$ with 
 additional ancillary series so long as they are helpful for forecasting.  For real-time
 estimation problems, we seek to determine the best possible estimates of a
 target given a relevant collection of ancillary series.

More formally, the real-time estimation problem is concerned with
projecting the target $Y_t$ onto the available data $X_{t:} = \{ X_t, X_{t-1},
\ldots \}$, i.e., the semi-infinite past.  This formulation presumes that
 we have access to relevant ancillary series, and that we have access to all
 present and past values.  In practice, databases only extend back a few decades,
 and the infinitely remote past represents merely an idyll useful for 
 mathematical simplicity.  The linear estimation problem seeks a linear estimate
  of the form
\[
   \widehat{Y}_t = \sum_{\ell \geq 0} \widehat{\psi} (\ell) \, X_{t-\ell}
     = \widehat{\Psi} (L) \, X_t,
\]
 which shows that we seek a linear
(time-invariant) concurrent filter $\widehat{\Psi} (L)$, applied to $\{ X_t \}$. 
 We desire that the
error in approximating the target with the available data be small with respect to MSE.
 If $\{ X_t \}$ were Gaussian, we could view our estimate as the conditional
expectation $\widehat{Y}_t = \EE [ Y_t \vert X_{t : } ]$, with the coefficients
$\{ \widehat{\psi} (\ell) \}$ selected to minimize the MSE
  of the approximation error $Y_t - \widehat{Y}_t$.  However, in our treatment in this book
 we do not presume Gaussian structure, and are not concerned with conditional expectations
 {\em per se}; rather, we seek linear solutions with minimal MSE.  
 

\begin{Definition} \rm
\label{def:lpp}
 The {\bf Linear Prediction Problem} (LPP) seeks the minimal
 MSE linear estimate that solves the real-time estimation problem
  arising from a scalar target.  That
 is, the LPP involves determining causal $\widehat{\Psi} (L)$ such that the
 prediction error
\[
 Y_t - \widehat{Y}_t = \left[ \Psi (L) - \widehat{\Psi} (L) \right] \, X_t
\]
 has mean zero and minimal MSE.
\end{Definition}

\begin{Example} {\bf Multi-step Ahead Forecasting.}   \rm
\label{exam:multi-step.fore.2}
The LPP corresponds to
  optimal $h$-step
 forecasting, and the forecast error is $[L^{-h} \,
 e_1^{\prime} - \widehat{\Psi} (L) ] \, X_t$.
  Note that although $\Psi (L)$ only involves one component series $\{ X_{t,1} \}$,
  the real-time concurrent filter $\widehat{\Psi} (L)$ can involve all $n$ component series.
\end{Example}

\begin{Example} {\bf HP Low-pass.}   \rm
\label{exam:hp-high.2}
  The LPP attempts to determine an optimal real-time trend estimate, where the target trend
 -- sometimes called the historical trend -- is defined through the HP low-pass filter.
  Here, both the target filter $\Psi (L)$ and the real-time 
   concurrent filter $\widehat{\Psi} (L)$   involve all $n$ component series.
\end{Example}




\subsection{Solution to the Linear Prediction Problem}

 When the data process is itself causal and linear, it is possible
 to give an explicit solution to the LPP in terms of the Wold
 decomposition (Brockwell and Davis, 1991).
  All purely nondeterministic weakly stationary (mean zero) processes
 have a Wold decomposition $X_t = \Theta (L) \epsilon_t$, where $\{
 \epsilon_t \}$ is $\mbox{WN} (\Sigma)$ and $\Theta (L) =
 \sum_{\ell \in \ZZ} \theta (\ell) \, L^{\ell}$.
 When $\theta (\ell) = 0 $ for all $\ell < 0 $, the process is called {\em causal}.
 First, the error in the LPP is denoted $E_t = Y_t - \widehat{Y}_t$, which is
 clearly mean zero  and covariance  stationary, in fact having spectral representation
\begin{equation}
\label{eq:dfa-error}
  E_t = \int_{-\pi}^{\pi} e^{i \omega t } \, 
   \left[ \Psi (z) - \widehat{\Psi} (z) \right] \, \mathcal{Z} (d\omega).
\end{equation}
  With these preliminaries, we can state the solution to the LPP.

\begin{Proposition}
 \label{prop:GPP}
 Suppose that $\{ X_t \}$ is mean zero and weakly stationary 
 with spectral representation (\ref{eq:specRep}), and moreover is
 causal, expressed as $X_t = \Theta (L) \, \epsilon_t$.    Then the solution
 to the LPP posed by a scalar target $Y_t = \Psi (L) \, X_t$ is given by
\begin{equation}
 \label{eq:GPPsoln}
 \widehat{\Psi} (L) = \sum_{\ell \geq 0 } \psi (\ell) \,
 L^{\ell} + \sum_{\ell < 0 } \psi (\ell)
 \,  { [ \Theta (L) ]}_{-\ell}^{ \infty  } \, L^{\ell} \, {\Theta (L) }^{-1}.
\end{equation}
 Moreover, the minimal MSE is given by
\begin{equation} 
\label{eq:minimalMSE}
 \frac{1}{ 2 \pi} \int_{-\pi}^{\pi}   \sum_{\ell > 0 } \psi (-\ell) \,
 z^{-\ell} {[ \Theta  (z) ]}_0^{\ell-1}   \,  \Sigma \,
  { {[ \Theta  (z) ]}_0^{ \ell-1} }^*  \,
   \sum_{\ell > 0 } \psi (-\ell) \,  z^{\ell}  \, d\omega.
\end{equation}
 \end{Proposition}

\paragraph{Proof of Proposition \ref{prop:GPP}.}
 In order for a linear solution to be MSE optimal, it is sufficient that the
 resulting error process be uncorrelated with the data $X_{t:}$.
   If we can show that the real-time signal extraction error process $\{ E_t \}$
  depends only on future innovations, then by the causality of $\{ X_t \}$ the error process 
  must be uncorrelated   with $X_{t:}$, establishing optimality.  
 The filter error of the putative solution is  
  given by
\begin{align*}
 \Psi (L) - \widehat{\Psi} (L) & = \sum_{\ell < 0 } \psi (\ell) \, L^{\ell} \,
   \left( 1 -   {[ \Theta (L) ]}_{-\ell}^{\infty} \, { \Theta (L) }^{-1} \right) \\
  & =  \sum_{\ell < 0 } \psi (\ell) \, L^{\ell} \, 
  {[ \Theta (L) ]}_{0}^{ -(\ell + 1)} \, { \Theta (L) }^{-1}.
\end{align*}
 Applying this to $\{ X_t \}$ yields
\[
  E_t = \sum_{\ell =1 }^{\infty} \psi (-\ell) \, {[ \Theta (L) ]}_0^{\ell - 1} \, 
   \epsilon_{t + \ell }.
\]
  Noting that ${[ \Theta (L) ]}_0^{\ell - 1}$ is an order $\ell-1$ polynomial in $L$,
 and is applied to $\epsilon_{t+ \ell}$, it is apparent that $E_t$ is a linear function
 of future innovations $\{ \epsilon_{t+1}, \epsilon_{t+2}, \ldots \}$.  Computing
 the variance of $E_t$ yields the expression for the minimal MSE.  $\quad \Box$


\begin{Remark} \rm  
\label{rem:GPPsoln}
 The formula (\ref{eq:minimalMSE}) gives us a lower
 bound  on the MSE when we use sub-optimal proxies for $\widehat{\Psi} (L)$.
\end{Remark}

 As indicated by Remark \ref{rem:GPPsoln}, the result of Proposition \ref{prop:GPP}
 is chiefly useful when we know $\Theta (L)$.  However, this is rarely the case in 
 practice: a classical parametric approach involves formulating a time series model, fitted 
 using the Gaussian likelihood, and finally computing the LPP solution in terms of
 the fitted model.  Alternatively, one might consider fitting a specified model such that
 the LPP MSE is minimized.  A more broad nonparametric approach involves 
 considering classes of concurrent filters and directly minimizing the LPP MSE over
 this class -- this is the methodology of Direct Filter Analysis (DFA).

\begin{Illustration} {\bf  VAR(1).} \rm
\label{ill:var1}
 Consider an LPP where the true process $\{ X_t \}$
 is a Vector Autoregression (VAR) of order 1.  This process can be described via
\begin{equation}
\label{eq:var1-def}
  X_t = \Phi \, X_{t-1} + \epsilon_t
\end{equation}
 for a matrix $\Phi$ that is stable, i.e., has 
 all eigenvalues bounded by one in modulus (L\"utkepohl, 2007). 
It is known that the VAR(1) has the causal representation  
 $\Theta (L) = {(1 - \Phi \, L )}^{-1}$.  Because
 for $\ell < 0$ 
\[
   { [ \Theta (L) ]}_{-\ell}^{ \infty  } = \sum_{j = -\ell}^{\infty} \Phi^j \, L^j 
  = \Phi^{-\ell} \, L^{-\ell} \, {( 1 - \Phi \, L )}^{-1},
\]
 we find that (\ref{eq:GPPsoln}) reduces to
\[
  \widehat{\Psi} (L) =   \sum_{\ell \geq 0 } \psi (\ell) \, L^{\ell} +
	\sum_{\ell < 0} \psi (\ell) \, \Phi^{-\ell}.
\]
  The second term in this expression we denote by $A_{\Psi} (\Phi)$.  Hence, the optimal
 concurrent filter is determined by applying the filter to past data and modifying the
 present weight $\psi (0)$ by adding the quantity $A_{\Psi} (\Phi)$.  In the case of
 $h$-step ahead forecasting of the first time series (Example 1), $\widehat{\Psi} (L)
 = A_{\Psi} (\Phi) = e_1^{\prime} \, \Phi^h$. 
 
 This formula demonstrates that it is 
 essential that $\Phi$ be stable, and if fitting a VAR(1) we must parametrize $\Phi$
 such that stability is guaranteed.  We describe a parameterization based upon
 Ansley and Newbold (1979) and Ansley and Kohn (1986), as summarized by Heaps (2020),
  which  provides a bijection from $\RR^{n^2} $
 to the space of stable $n \times n $ matrices.
 For any positive definite symmetric matrix $\Sigma$, the Generalized Cholesky Decomposition
  (GCD) takes the form $\Sigma = L \, D \, L^{\prime}$, where
    $L$ is  unit lower triangular and $D$ is diagonal with positive entries --
    see Golub and Van Loan (1996).
    In such as case, we let $\Sigma^{1/2}$ denote $L \, D^{1/2}$, which is lower triangular
    with non-unit positive entries on the diagonal.
  Then the forward mapping takes a matrix $A$ with arbitary real entries to
  a matrix $\Phi$ with eigenvalues of magnitude less than one:
\begin{align*}
  A & \tends \Phi \\
  P & = { \left( { \left[  1_n + A \, A^{\prime} \right] }^{1/2} \right)}^{-1} \, A \\
  S & = { \left( { \left[ 1_n - P \, P^{\prime} \right] }^{1/2} \right) }^{-1} \\
  \Phi & = S \, P \, S^{-1}.
\end{align*}
    The inverse transformation is  given by
\begin{align*}
   \Phi & \tends A \\
  \mbox{vec} \Gamma & = { \left( 1_{n^2} - \Phi \otimes \Phi \right) }^{-1} \, 
  \mbox{vec} 1_n \\
  S & = \Gamma^{1/2} \\
  P & = S^{-1} \, \Phi \, S \\
  A & = { \left( { \left[ 1_n - P \, P^{\prime} \right] }^{1/2} \right) }^{-1} \, P.
\end{align*}
\end{Illustration}




\section{Model Fitting via LPP MSE Minimization}
%\label{gppmf}

 Here we study the mechanics of fitting a parametric model such that the LPP MSE
 is minimized.  In the case of the one-step ahead forecasting MSE, this is related
 to Whittle estimation of vector time series models (c.f.,  Taniguchi and Kakizawa (2000)).
  We will focus on the class of separable causal linear models, wherein the innovation variance
 $\Sigma$ is governed by a separate set of parameters from those describing the power
 series $\Theta (L)$.  The model is essentially described through a particular class of
 power series   $\Theta_{\vartheta} (L)$, parameterized by a vector $\vartheta $
  belonging to some model  parameter manifold.    Hence the model sdf is
\[
  F_{\vartheta} (\omega) = \Theta_{\vartheta} (z) \, \Sigma \,  { \Theta_{\vartheta} (z) }^*.
\]
 However, the model may be misspecified: the process' sdf is denoted $\widetilde{F}$, 
  and may not
 belong to the model class.  The goal of model fitting is to determine $\vartheta$ 
  such that
 $F_{\vartheta}$ is a good approximation to $\widetilde{F}$.  
 Clearly, knowing $\vartheta$ does not
 fully determine $F_{\vartheta}$ because $\Sigma$ remains unknown; however, the methods
 described below provide for estimates of $\Sigma$ in terms of $\vartheta$ and the process.
 From the proof of Proposition \ref{prop:GPP} we know that the filter error satisfies
\[
  E_t = \sum_{\ell =1 }^{\infty} \psi (-\ell) \, L^{-\ell} \,  {[ \Theta (L) ]}_0^{\ell - 1} \, 
   { \Theta_{\vartheta} (L) }^{-1} \, X_t.
\] 
 In other words, we have an error filter $ \Xi_{\vartheta} (L) \,  
  { \Theta_{\vartheta} (L) }^{-1}  $,
 where
\[
  \Xi_{\vartheta} (L)  = \sum_{\ell =1 }^{\infty} \psi (-\ell) \,  L^{-\ell} \,
   {[ \Theta (L) ]}_0^{\ell - 1},
\]
  such that for any choice of $\vartheta $ we can compute filter errors $\{ E_t \}$.
 Note that these are not in general to be interpreted as residuals, and they need not be white noise.
  But we can seek to minimize their variance.  In practice,
 the calculation of such filter errors  may require a truncation of the error filter, because the finite
 sample $X_1, X_2, \ldots, X_T$ is available, not the entire infinite past.  The error filter is
 $1 \times n$, and for any $\vartheta $ and any Hermitian function $G$ we can compute
\begin{align*}
  J_{\Psi} (\vartheta, G) & 	= \frac{1}{2 \pi} \, \int_{-\pi}^{\pi} \Xi (z) \,
  { \Theta_{\vartheta} (z) }^{-1} \, G(\omega)
  \,  {{ \Theta_{\vartheta} (z) }^{-1} }^* \, { \Xi (z) }^* \, d\omega  = \mbox{tr} \{ 
 { \langle G \, K_{\vartheta} \rangle }_0 \} \\
   K_{\vartheta} (z) & =  {{ \Theta_{\vartheta} (z) }^{-1} }^* \, { \Xi (z) }^* \,  \Xi (z) \, 
 { \Theta_{\vartheta} (z) }^{-1}.
\end{align*}
 Then $\mbox{Var} [ E_t ] = J_{\Psi} (\vartheta, \widetilde{F})$.  As we seek to
 minimize the variability in the filter errors,
 we can take $J_{\Psi} (\vartheta, \widetilde{F})$ as our criterion function. 
 However, this will only determine the proximity of the model to the true sdf,
 which is unavailable to us -- in order to compute actual
 parameter estimates, we must utilize the data to approximate the true sdf. 
 There are basic results giving asymptotic normality for simple functionals 
 of the periodogram, and therefore this crude  estimator of the true sdf is sufficient 
 for our purposes.  We propose 
$J_{\Psi} (\vartheta, \widehat{F})$ as
 an estimator of $J_{\Psi} (\vartheta, \widetilde{F})$, and intend that the respective minimizers have the same relationship.
 Namely, if $\vartheta (\widetilde{F}) $ is the unique minimizer of 
 $J_{\Psi} (\vartheta, \widetilde{F}) $ and
  $\vartheta (\widehat{F}) $ is the unique minimizer of $J_{\Psi} (\vartheta, \widehat{F}) $, then
 $\vartheta (\widehat{F})$ is an estimator of $\vartheta (\widetilde{F})$, which is called the {\em pseudo-true value} (PTV).
  From the PTV and estimator, we can also compute the innovation covariance matrix by the formulas
\begin{align*}
  \Sigma (\widetilde{F}) & = { \langle { \Theta_{\vartheta (\widetilde{F} ) } (z) }^{-1} \,
   \widetilde{F} \, 
 { { \Theta_{\vartheta(\widetilde{F}) } (z) }^{-1} }^*
	\rangle }_0 \\
    \Sigma (\widehat{F}) & = { \langle { \Theta_{\vartheta (\widehat{F} ) } (z) }^{-1} \, \widehat{F} \, 
{ { \Theta_{\vartheta(\widehat{F}) } (z) }^{-1} }^*
	\rangle }_0.
\end{align*}
 In the special case that the model is correctly specified, there exists some 
$\widetilde{\vartheta}$ and $\widetilde{\Sigma}$ such that
 $\widetilde{F} (\omega) = \Theta_{\widetilde{\vartheta}} (z) \, \widetilde{\Sigma} \, 
{\Theta_{\widetilde{\vartheta}} (z)  }^*$; it is shown below
  that the PTV matches the truth.
 
\begin{Proposition}
 \label{prop:PTV-truth}
 Given an LPP $\Psi$ and the criterion function $J_{\Psi} (\vartheta, \widetilde{F})$,
 if the model is correctly specified and the 
 minimizer $\vartheta (\widetilde{F})$ is unique then it equals the true parameter 
$\widetilde{\vartheta}$, and $\Sigma (\widetilde{F}) = \widetilde{\Sigma}$.
\end{Proposition}

\paragraph{Proof of Proposition \ref{prop:PTV-truth}.}
 Because the model is correct, the criterion function becomes
\[
  J_{\Psi} (\vartheta, \widetilde{F}) = { \langle \Xi (z) \, { \Theta_{\vartheta   } (z) }^{-1} \, 
   \Theta_{\widetilde{\vartheta}} (z) \, \widetilde{\Sigma} \, 
  {\Theta_{\widetilde{\vartheta}} (z)  }^* \, { { \Theta_{\vartheta  } (z) }^{-1} }^*
 \, { \Xi (z) }^* \rangle }_0,
\]
 which for $\vartheta = \widetilde{\vartheta}$ achieves the minimal value:
\[
  J_{\Psi} (\widetilde{\vartheta}, \widetilde{F})  ={ \langle \Xi (z) \, 
  \widetilde{\Sigma} \,    { \Xi (z) }^* \rangle }_0.
\]
  Because the minimizer is unique by assumption, 
  $\vartheta (\widetilde{F}) = \widetilde{\vartheta} $. 
  Plugging this into the formula
 for $\Sigma (\widetilde{F})$, we see that it equals $\widetilde{\Sigma}$.  $\quad \Box$

\vspace{.5cm}


\begin{Remark} \rm  
\label{rem:PTVunique}
 One of the conditions of Proposition \ref{prop:PTV-truth} is that the PTV is unique.  
 Non-uniqueness can arise in practice, but any PTV that minimizes the LPP criterion 
 is suitable for generating an optimal solution.  Hence, even though in this situation
 a particular PTV does not equal the true parameter, yet it generates the same minimal
 value of $J_{\Psi}$ as the true parameter, and is ``just as good" as the true parameter
 for purposes of the particular LPP.  We do not care about correct model fitting {\it per se}.
\end{Remark}


\begin{Example} {\bf Multi-step Ahead Forecasting.}  \rm
\label{exam:multi-step.fore.3}
 For $h$-step ahead forecasting, only $\psi (-h)$
  is nonzero, so that
 $\Xi (L) =  L^{-h} \,  e_1^{\prime} \, {[ \Theta (L) ]}_0^{ h - 1}$. 
  Hence, the criterion function $J_{\Psi} $ fits models so as to minimize
 (in the frequency domain)  $h$-step ahead forecast error of the first series. 
 In the special case that $h=1$, the criterion function is
\[
 J_{\Psi} (\vartheta, G) =  e_1^{\prime} \, { \langle   { \Theta_{\vartheta   } (z) }^{-1}  \,
  G \, { { \Theta_{\vartheta  } (z) }^{-1} }^*  \rangle }_0 \, e_1.
\]
   If we were to compute such a measure for all $n$ series, and sum over the $n$ criteria, we would obtain the concentrated Whittle 
 likelihood, namely
\[
  \mbox{tr} \, \{  { \langle   { \Theta_{\vartheta   } (z) }^{-1}  \, G \,
  { { \Theta_{\vartheta  } (z) }^{-1} }^*  \rangle }_0 \}.
\]
  See the discussion in McElroy and Findley (2015).  This connection justifies viewing $J_{\Psi}$ as a generalization of the Whittle
 likelihood from one-step ahead forecasting to more general real-time LPPs.
\end{Example}


%\vspace{.5cm}

It is possible to conduct inference from the PTVs on the basis of the estimates 
$\vartheta (\widehat{F})$, and thereby assess model fit.  
 In order to formulate our result, we assume that the PTVs are not on the
 boundary of the parameter set (otherwise the limit theory
 is non-standard; cf. Self and Liang (1987)), and  that they are unique. 
We also assume that the  Hessian  $H(\vartheta) = \nabla \nabla^{\prime}
  J_{\Psi} (\vartheta, \widetilde{F}) $  of $J_{\Psi} $ is positive definite at the PTV.
  The so-called Hosoya-Taniguchi (HT) conditions of Hosoya and Taniguchi (1982) 
impose sufficient regularity   on the process
 $\{ X_t \}$ for our purposes; these conditions require that $\{ X_t \}$ is a 
causal filter of a higher-order martingale difference.  
 A simpler limiting variance expression is available if the fourth order cumulant
 function of $\{ X_t \}$ is zero.  

\begin{Theorem}
\label{thm:LPP-Clt}
 Suppose that $\vartheta (\widetilde{F}) $ exists uniquely in the interior of the model 
 parameter space,   and that $H(\vartheta (\widetilde{F}))$ is 
 positive definite.  Suppose that $\{ X_t \}$ has finite fourth moments, conditions (HT1)-(HT6) of Taniguchi and Kakizawa (2000, pp.55-56)
 hold, and that the fourth order cumulant function of $\{ X_t \}$ is zero.  Then the estimator is consistent for the PTV, and
\[
 \sqrt{T} \, \left( \vartheta( \widehat{F} ) - \vartheta (\widetilde{F}) \right) 
 \convinlaw \mathcal{N} \left( 0, { H(\vartheta (\widetilde{F})) }^{-1} \, V (\vartheta (\widetilde{F})) \, { H(\vartheta (\widetilde{F})) }^{-1} \right)
\]
 as $T \tends \infty$, where 
\[
  V_{jk} (\vartheta) =  \mbox{tr} \{ { \langle  \partial_j K_{\vartheta} (z) \, \widetilde{F} \, \partial_k K_{\vartheta} (z) \, \widetilde{F} \rangle }_0 \}.
\]
\end{Theorem}

\paragraph{Proof of Theorem \ref{thm:LPP-Clt}.}
 A Taylor series exapansion of the gradient of $J_{\Psi} (\vartheta, \widehat{F})$ and $J_{\Psi} (\vartheta, \widetilde{F})$ yields the 
 asymptotic expression
\[
  \sqrt{T} \, \left( \vartheta( \widehat{F} ) - \vartheta (\widetilde{F}) \right)  = o_P (1) 
   - { H (\vartheta( \widetilde{F} )) }^{-1} \,  \mbox{tr} \, \{  {  \langle (\widehat{F} - \widetilde{F}) \, \nabla K_{\vartheta} \rangle }_0 \},
\]
 where the trace operator acts upon the spectral matrices, for each component 
 of the gradient operator.  Our assumptions allow us to apply Lemma 3.1.1 of Taniguchi
 and Kakizawa (2012) to the right hand expression, yielding the stated central limit theorem.
 $\quad \Box$

\vspace{.5cm}

\begin{Illustration} {\bf  VAR(1).}  \rm
\label{ill:var1.2}
When the model is a VAR(1), the parameter vector $\vartheta$
 describes the entries of $\Phi$ in such a way that the matrix is stable,
 as described previously.  The error filter can be 
 expressed
\begin{align*}
    \Xi (L) & =   \sum_{\ell =1 }^{\infty} \psi (-\ell) \,  L^{-\ell} \,
   \sum_{k=0}^{\ell-1} \Phi^k \, L^k \\
  & =  \sum_{\ell =1 }^{\infty} \psi (-\ell) \,  L^{-\ell} \,
   \left( 1 - \Phi^{\ell} \, L^{\ell} \right) \, {( 1 - \Phi \, L )}^{-1} \\
 & =  \left( \sum_{\ell =1 }^{\infty} \psi (-\ell) \,  L^{-\ell} -
    A_{\Psi} (\Phi) \right)  
    \, {( 1 - \Phi \, L )}^{-1}.
\end{align*}
  It follows that 
\begin{align*}
 J_{\Psi} (\vartheta, G) & = \sum_{\ell,k > 0} \psi (-\ell) \, {\langle G \rangle}_{\ell-k} \, 
	{\psi (-k)}^{\prime} - A_{\Psi} (\Phi) \, \sum_{k > 0} {\langle G \rangle}_{-k} 
	\, {\psi (-k)}^{\prime}  \\
  & - \sum_{\ell > 0} \psi (-\ell) \, { \langle G \rangle }_{\ell} \, 
	{ A_{\Psi} (\Phi) }^{\prime} + { A_{\Psi} (\Phi) }^{} \, { \langle G \rangle }_0 \,
 { A_{\Psi} (\Phi) }^{\prime},
\end{align*}
 which is easily computed.  Optimization with respect to $\vartheta$ is straightforward.
 In the special case of $h$-step ahead forecasting, the criterion further simplifies to
\begin{align*}
  J_{ L^{-h}} (\vartheta, G) & = 
  e_1^{\prime} \, { \langle G \rangle }_0 \, e_1 - e_1^{\prime} \, \Phi^h \, { \langle G \rangle}_{-h} 
	\, e_1 \\
 & - e_1^{\prime} \, {\langle G \rangle }_h \, {\Phi}^{h \prime} \, e_1
	+ e_1^{\prime} \, \Phi^h \, { \langle G \rangle }_0 \, {\Phi}^{h \prime} \, e_1,
\end{align*}
 and any $\vartheta$ such that
\[
  e_1^{\prime} \, \Phi^h = e_1^{\prime} \, { \langle G \rangle }_h \, { \langle G \rangle }_0^{-1}
\]
 is a critical point.  If the model is correctly specified, then setting $G$ equal to the spectral
 density of the VAR(1) yields a minimal possible MSE of
\[
   e_1^{\prime} \, \left( \Gamma (0) - \Phi^h \, \Gamma (0) \, {\Phi}^{h \prime} \right) \, e_1.
\]
 It is known that Whittle estimation corresponds to the $h=1$ case, 
 with a criterion function given
 by the determinant of the forecast MSE matrix (McElroy and Findley, 2015), and in essence
 incorporates forecast error from all $n$ series.  In the above, 
 the criterion only depends on the
 performance of the first series, which allows a practitioner to focus on parameter values
 that sacrifice performance on the other $n-1$ series in order to achieve superior results for
 the first series.
\end{Illustration}



\begin{Exercise} {\bf Correct VAR(1) LPP.} \rm
\label{exer:var1lpp-correct}
 Simulate a sample of size $T=100$ from a
  bivariate VAR(1) process with 
\[
  \Phi = \left[ \begin{array}{cc} 1 & 1/2 \\ -1/5 & 3/10 \end{array} \right]
\]
 and $\Sigma$ equal to the identity.  The eigenvalues are $4/5$ and $1/2$.
 Then utilize the LPP criterion $J_{\Psi} (\vartheta, G)$ to fit a VAR(1) model (use
  the stable parametrization of $\Phi$, using two choices of $\delta$) with both 
 the $2$-step ahead forecasting LPP (Example \ref{exam:multi-step.fore}) and the 
 ideal low-pass  LPP (Example \ref{exam:ideal-low}) with 
  $\mu = \pi/24$, where $G$ is given by the periodogram of the sample.
 (As usual, the first of the two series is the target.)
 Do you obtain a unique minimizer?    Do the parameter estimates appear to be consistent? 
  How do the estimates compare to the Yule-Walker estimates?
 Repeat for $T=200$ and $T=500$.
\end{Exercise}

\begin{Schunk}
\begin{Sinput}
> # Simulate a Gaussian VAR(1) of sample size 100:
> set.seed(1234)
> T <- 100
> phi.matrix <- rbind(c(1,.5),c(-.2,.3))
> innovar.matrix <- diag(2)
> true.psi <- var.par2pre(array(phi.matrix,c(2,2,1)))
> gamma <- VARMAauto(array(phi.matrix,c(2,2,1)),NULL,innovar.matrix,10)
> gamma.0 <- gamma[,,1]
> x.init <- t(chol(gamma.0)) %*% rnorm(2)
> x.next <- x.init
> x.sim <- NULL
> for(t in 1:T)
+ {
+ 	x.next <- phi.matrix %*% x.next + rnorm(2)
+ 	x.sim <- cbind(x.sim,x.next)
+ }
> x.sim <- ts(t(x.sim))
> x.acf <- acf(x.sim,type="covariance",plot=FALSE,lag.max=T)[[1]]
> x.acf <- aperm(aperm(x.acf,c(3,2,1)),c(2,1,3))
> # Yule-Walker fit
> phi.yw <- x.acf[,,2] %*% solve(x.acf[,,1])	# phi coefficient from YW
> yw.psi <- var.par2pre(array(phi.yw,c(2,2,1)))
> # 1-step ahead forecasting
> psi.array <- array(0,c(1,2,1))
> psi.array[,,1] <- c(1,0)
> acf.array <- x.acf[,,1:3]
> theta <- rnorm(4)
> var1.fit.1step <- optim(theta,lpp.var1,psi.array=psi.array,
+ 	acf.array=acf.array,method="BFGS")
> # fit output
> print(var1.fit.1step$value)		
\end{Sinput}
\begin{Soutput}
[1] 1.023043
\end{Soutput}
\begin{Sinput}
> print(lpp.var1(yw.psi,psi.array,acf.array))
\end{Sinput}
\begin{Soutput}
         [,1]
[1,] 1.023043
\end{Soutput}
\begin{Sinput}
> # phi coefficient from 1-step LPP 
> print(var.pre2par(var1.fit.1step$par,1,2))	
\end{Sinput}
\begin{Soutput}
, , 1

           [,1]      [,2]
[1,]  1.0297320 0.6310257
[2,] -0.1752084 0.7950063
\end{Soutput}
\begin{Sinput}
> # phi coefficient from Yule-Walker
> print(phi.yw)
\end{Sinput}
\begin{Soutput}
           [,1]      [,2]
[1,]  1.0297332 0.6310298
[2,] -0.1719921 0.2678559
\end{Soutput}
\begin{Sinput}
> # 2-step ahead forecasting
> psi.array <- array(0,c(1,2,2))
> psi.array[,,1] <- c(0,0)
> psi.array[,,2] <- c(1,0)
> acf.array <- x.acf[,,1:3]
> theta <- rnorm(4)
> var1.fit.2step <- optim(theta,lpp.var1,psi.array=psi.array,
+ 	acf.array=acf.array,method="BFGS")
> # fit output
> print(var1.fit.2step$value)		
\end{Sinput}
\begin{Soutput}
[1] 2.674102
\end{Soutput}
\begin{Sinput}
> print(lpp.var1(yw.psi,psi.array,acf.array))
\end{Sinput}
\begin{Soutput}
         [,1]
[1,] 2.680453
\end{Soutput}
\begin{Sinput}
> # phi coefficient from 2-step LPP 
> print(var.pre2par(var1.fit.2step$par,1,2))	
\end{Sinput}
\begin{Soutput}
, , 1

            [,1]       [,2]
[1,]  1.00935074  1.0897652
[2,] -0.06643582 -0.3207922
\end{Soutput}
\begin{Sinput}
> # phi coefficient from Yule-Walker
> print(phi.yw)
\end{Sinput}
\begin{Soutput}
           [,1]      [,2]
[1,]  1.0297332 0.6310298
[2,] -0.1719921 0.2678559
\end{Soutput}
\begin{Sinput}
> # low-pass LPP
> mu <- pi/24
> psi.array <- array(c(1,0) %x% sin(seq(1,T-1)*mu)/(pi*seq(1,T-1)),c(1,T-1,2))
> psi.array <- aperm(psi.array,c(1,3,2))
> acf.array <- x.acf
> theta <- rnorm(4)
> var1.fit.bk <- optim(theta,lpp.var1,psi.array=psi.array,
+ 	acf.array=acf.array,method="BFGS")